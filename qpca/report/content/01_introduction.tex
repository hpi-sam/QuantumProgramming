%1st Paragraph expands the context by supporting it with references
% \enquote{Quantum computing has become a reality. Quantum computers are available to everybody via cloud service or simulation.}(\cite{Ebert, 2021})(restrictions apply - talk with chris first)

%%2nd Paragraph: Narrow the motivation to a technological challenge.
Requirements for e.~g. robust, scale-able, or flexible solutions apply to classical\footnote{In this paper the term \emph{classical} is sued to refer to computer systems working with bits that represent exactly one out of two possible states \{0, 1\}.} and quantum based software alike. Hence, the engineering challenge is to adapt best practices in software engineering for the quantum world.

%%3rd Paragraph: Briefly list the existing technological solutions
A survey through the landscape and horizons of quantum software engineering has been compiled by J. Zhao \cite{Zhao_2020} and they introduced a quantum software life-cycle from requirements analysis up to maintenance based on the classical waterfall model. They also discuss various challenges in quantum computing that have been addressed and solved theoretically by the research community already.

%%4th Paragraph: Point to the gap between the current solutions/knowledge and the technical challenge described in paragraph 2

%%5th Paragraph: My insight
In this paper the problem of feature reduction on high dimensional data sets with the help of \emph{Principal Component Analysis} (PCA) is used as an example to apply the full quantum software life-cycle. This challenge is approached from the perspective of a developer transitioning from classical into quantum programming.

%%6th Paragraph: State the problem and explain it in more detail
\emph{The question} that is investigated is up to which degree current tools and techniques proposed by the research community can be directly applied to build a quantum based PCA solution. For example, in \cite{He_2021} or \cite{Daski_2015} the authors stated that transforming a classical data set into a quantum state is not necessarily straight-forward to do. They do give a few numerical examples but the (mathematical) computation of the unitary operators $U_prep$ to prepare a corresponding quantum state are in respect to already know eigenvalues.

%%7th Paragraph: Detail your approach
\emph{The approach} taken in this paper briefly goes through the different steps in the quantum software life-cycle and applies them to implement a simple quantum based PCA solution for $2 \times 2$ feature matrices. In a second step this solution is enhanced by exchanging parts of the algorithm to overcome the limitations of $2 \times 2$ feature matrices. The integration step will be encapsulated to keep it indendent from the actual PCA algorithm as it will turn out to stay a challenge within this papers project phase to integrate arbitrary ($2 \times 2$) feature matrices.

%%8th Paragraph: Detail your contributions
\emph{The contributions} comprise a set of challenges still to take before the in this paper evaluated quantum based solutions can fully generalize to various software qualities\footnote{In this paper the terms \emph{software qualities} is prefered but can be used synonymously with \emph{non-functional requirements} or \emph{software structural qualities}.}, especially for developers who are new to the different concepts in quantum mechanics or quantum programming. Taking into account the level engineering communities like the ones behind \emph{qiskit} or \emph{cirq} have been catching up with proposals from the research community already, the \emph{implications} of this papers results will be subject to constant (re-)evaluation as well. To allow \emph{reproduction} the source code and test-data are publicly available to the community. \cite{github_repo}

The rest of the paper is structured as follows. In section ~\ref{sec:background} relevant concepts to follow this paper are introduced namely a brief overview on PCA in subsection ~\ref{subsec:pca} and different methods to initialize data into and extract data from quantum states in subsection ~\ref{subsec:integration}. In section ~\ref{sec:survey} the different steps to implement two simple quantum based PCA solutions are described by following the phases of a quantum software engineering life-cycle. The results and challenges are discussed in ~\ref{sec:discussion}. Finally, section ~\ref{sec:conclusion} summarizes this papers contributions and future work.
