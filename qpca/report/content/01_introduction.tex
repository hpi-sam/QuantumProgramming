%1st Paragraph expands the context by supporting it with references
% \enquote{Quantum computing has become a reality. Quantum computers are available to everybody via cloud service or simulation.}(\cite{Ebert, 2021})(restrictions apply - talk with chris first)

%%2nd Paragraph: Narrow the motivation to a technological challenge.
Requirements for e.~g. robust, scale-able, or flexible solutions apply to classical and quantum based software alike. Hence, the engineering challenge is to adapt best practices in software engineering for the quantum world.

%%3rd Paragraph: Briefly list the existing technological solutions
A survey through the landscape and horizons of quantum software engineering has been compiled by J. Zhao \cite{Zhao_2020} and they introduced a quantum software life-cycle from requirements analysis up to maintenance based on the concepts from classical computing. They also discuss various challenges in quantum computing that have been addressed and solved theoretically by the research community already.

%%4th Paragraph: Point to the gap between the current solutions/knowledge and the technical challenge described in paragraph 2
Nonetheless, applying the full proposed quantum software life-cycle on a hybrid architecture, embedding quantum solutions inside of a classical framework, is less well researched.

%%5th Paragraph: My insight
Therefore, in this paper the problem of feature reduction on high dimensional data sets with the help of \emph{Principal Component Analysis} (PCA) is used as an example to apply the full quantum software life-cycle. This challenge is approached from the perspective of a developer transitioning from classical into quantum programming.

%%6th Paragraph: State the problem and explain it in more detail
\emph{The question} that is investigated is whether current tools and techniques proposed by the research community can be applied to build a robust, scale-able and flexible quantum PCA solution. These qualities imply the following questions:
\begin{description}
  \item [Robustness] How can the achievements in classical testing and test driven development be transferred to quantum programming?
  \item [Scalability] How can the implementation be platform-agnostic and able to scale up once larger amounts of qubits are available?
  \item [Flexibility] How can implementation details be exchanged flexible to ensure maintenance and openness to future changes?
\end{description}

%%7th Paragraph: Detail your approach
\emph{The approach} taken in this paper briefly goes through the different steps in the quantum software life-cycle and applies them to implement a simple quantum based PCA solution for $2 \times 2$ feature matrices. In a second step this solution is enhanced by exchanging parts of the algorithm to overcome the limitations of $2 \times 2$ feature matrices.

%%8th Paragraph: Detail your contributions
\emph{The contributions} comprise a set of challenges still to take before quantum programming can generalize the addressed software qualities when applying a full quantum software life-cycle. As quantum programming is a young discipline the tools and techniques are subject to constant changes. Therefore, the \emph{implications} of this papers results are subject to constant (re-)evaluation as well. To allow \emph{reproduction} the source code and test-data are publicly available to the community \cite{github_repo}.

The rest of the paper is structured as follows. In section \ref{sec:background} ... In section \ref{sec:survey} ... The results and discussion of the challenges are discussed in \ref{sec:discussion}. Finally, section \ref{sec:conclusion} summarizes this papers contributions and future work.
