\noindent
This section discusses the implementation process of a beginner-friendly version of a quantum PCA. The process itself is documented by surveying through the different phases of a quantum software life-cycle as proposed in \cite[pp. 9-10]{Zhao_2020} and can be seen in Figure ~\ref{fig:lifecycle}.

\begin{figure}
  \centering
  \includegraphics[width=\linewidth]{assets/survey_quantum_software_lifecycle.png}
  \caption{A quantum software life cycle. Figure taken from \cite{Zhao_2020}.}
  \label{fig:lifecycle}
\end{figure}

As a functional requirement the problem of feature reduction on high dimensional data sets was taken. In this case \enquote{high} dimensional included at least two dimensions and implied to not be restricted to it if the amount of available qubits of the underlying hardware allows it. As an example, the numerical examples of the quantum PCA implementations referenced in the introduction include $4 \times 4$ matrices as well. As for software qualities the support for the following three were investigated:
\begin{description}
  \item [Testability] Different test strategies have been surveyed by \cite[pp. 12-17]{Zhao_2020}. Can they be applied out of the box? Which support is given by the chosen library / libraries?
  \item [Scalability] How to scale up once larger amounts of qubits are available?
  \item [Flexibility] How can implementation details be exchanged or enhanced to ensure maintenance and openness to future changes?
\end{description}

The following two design decisions were made: (1) the PCA will be used to tackle the problem of feature reduction and (2) the algorithm discussed in \ref{subsec:qpca} will be used. The stated limitation to $2 \times 2$ feature matrices is a strong one with respect to the defined software qualities. The idea was to use this solution to identify and design test cases early in the implementation phase and to enhance the solution by exchanging the swap test with the \emph{Quantum Phase Estimation} (QPE) in a second step.

\begin{figure}
  \centering
  \includegraphics[width=\linewidth]{assets/survey_algorithm_main-ref19.png}
  \caption{A simple swap test based PCA. Figure taken from \cite{Lokho_2020}.}
  \label{fig:mainref19}
\end{figure}

The general layout of the swap test based solution can be seen in Figure ~\ref{fig:mainref19}. The authors of \cite[pp. 64-65]{Lokho_2020} suggested, to encode the quantum state via density operation and Schmidt decomposition. All necessary steps are part of the classical pre-processing. To interpret the measurements from the quantum circuit and to calculate the eigenvalues is part of the classical post-processing. To be able to exchange the swap test with QPE easily, both steps were encapsulated into a respective operation.

For implementation the following libraries were used: python, unittest, numpy, and cirq. As parts of the solution are classical, classical unit tests were applied for those steps respectively. The cirq library also provides support for unit testing with their \emph{cirq.testing} module. Especially the following two assertions were used multiple times during this project:
\begin{itemize}
  \item assert\_same\_circuits $\rightarrow$ tests if two circuits are equivalent to each other
  \item assert\_has\_diagram $\rightarrow$ tests the text representation
\end{itemize}

The cirq implementation of \emph{Quantum Fourier Transformation} \cite{Cirq_Qft} was used as a reference implementation. This decision was not ideal since it implements the swap test algorithm as a \emph{cirq.raw\_types.Gate} type and provides the corresponding operation as a wrapper function. Since gates only know qubits but not register of qubits, this papers swap test composition makes some assumptions on how qubits are entered into the gate.

For end-to-end testing the following two data sets were used:
\begin{itemize}
  \item PCA example from \cite[p. 63]{Lokho_2020}:
    \begin{itemize}
      \item feature set $X = (4, 3, 4, 4, 3, 3, 3, 3, 4, 4, 4, 5, 4, 3, 4)$
      \item feature set $Y = (3028, 1365, 2726, 2538, 1318, 1693, \\ 1412, 1632, 2875, 3564, 4412, 4444, 4278, 3064, 3857)$
    \end{itemize}
  \item Numerical example from \cite[p. 5]{He_2021}: $A = \begin{bmatrix} 1.5 & 0.5 \\ 0.5 & 1.5 \end{bmatrix}$
\end{itemize}
So far the test cases were run on a simulator only. This already introduced the challenge to handle a certain level of uncertainty in the measurement results. The actual result will only be almost equal to the expected result up to a certain delta. The cirq library also supports noise simulation but this was not yet integrated in this papers current test suite.

As an enhancement to the current solution the swap test operation was exchanged with QPE. This paper followed the algorithm proposed in \cite{Qtb_Qpe}. It was implemented as a \emph{cirq.raw\_types.Operation} type.

Exchanging the swap test with the QPE overcame the limitation to $2 \times 2$ feature matrices. With this change the solution also is more flexible with respect to further improvements already proposed by the research community. Take for instance the state-of-art quantum PCA algorithm proposed by \cite{He_2021}, it includes a quantum based singular value thresholding step to only compute the $n$ most relevant eigenvalues. Therefore, this enhancement supports the maintenance of the code but also highlights the tight relationship between the quantum parts of the solution and classical pre-processing and post-processing respectively.

