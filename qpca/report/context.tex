1. Quantum Computing

[yanofsky paper - begin]
\begin{description}
  \item[Quantum system] A quantum system is a system with its state and transitions not being deterministic. Furthermore, the probabilities of states and transitions are given as complex numbers $c$ such that $|c|^2$ is a real number between $0$ and $1$.
  \item[Qubits] In quantum computing the concept of bits is extended to the so called \emph{quantum bits} or \emph{qubits}. A \emph{qubit} can be any state that can be represented as:
    \[
      \begin{matrix} 0 \\ 1 \end{matrix}
      \begin{bmatrix} c_0 \\ c_1 \end{bmatrix}
      \text{, where }
      |c_0|^2 + |c_1|^2 = 1 \text{ and } c_0, c_1 \in \mathbb{C}
    \]
\end{description}
Note that from the previous introduced definition of a \emph{quantum system} the probabilities of states and transitions are complex numbers that should comply to the constrain that their square product leads to real number between $0$ and $1$. To perform these state transitions quantum operations are defined from quantum gates:
\begin{description}
\item[Quantum Gates] A quantum gate is any unitary matrix that manipulates qubits. [double check if unitary is the only required attribute]
\end{description}

-- Hadamard gate and R_y gate as examples -> will be referenced for the types of measurements
[yanofsky paper - end]

[qiskit textbook - begin]
A common notation is the so called \emph{bra ket} notation: [example]. To chain different gates to build more complex oprations the \emph{outer product} or \emph{matrix multiplication) is applied: [example].

To represent a quantum state the \emph{quantum state vector} or a \emph{density matrix} can be used. The \emph{ket} notation refers to a column state vector and the \emph{bra} notation to a transposed column state vector (row vector). Therefore, the density matrix can be noted as [ket bra notation here].

--- diff types of measurements (diff bases)
[qiskit textbook - end]

Complete randomness or increasment in computation speed are two goals commonly assigned with quantum computing. To achieve these goals the following concepts from quantum mechanics are applied:

[mooc begin]
\begin{description}
  \item[Superposition] With the given representation of a qubit it is possible to create a quantum state that is a combination of $|0\rangle$ and $|1\rangle$:
    \[
      \begin{bmatrix} c_1 \\ c_2 \end{bmatrix} = c_1 |0\rangle + c_2 |1\rangle
    \]
    If $c_1$ and $c_2$ are non-zero, the qubit's state contains both $|0\rangle$ and $|1\rangle$ at the same time which is called superposition. --> reference hadamard gate here.

  \item[Entanglement] With two qubits a quantum state can be represented as a combination of:
    \[
      c_1|00\rangle + c_2|01\rangle + c_3|10\rangle + c_4|11\rangle
    \]
    where $|c_1|^2 + |c_2|^2 + |c_3|^2 + |c_4|^2 = 1$, $c_1, c_2, c_3, c_4 \in \mathbb{C}$ and where $|01\rangle$ means the first qubit is in state $|0\rangle$ and the second qubit is in state $|1\rangle$. If two or more of the complex factors are non-zero it is not possible to separate the qubits.
[mooc - end]


2. Engineering
- goal
- how to approach

2.2 Circuit initialization techiques
-- bitwise
-- amplitude stuff
-- purifying / reduced density matrix


3. PCA
In this paper the \epmh{Principal Component Analysis} (PCA) will be used as a target problem to solve. The algorithm can be used on high-dimensional data to extract the $n$ most significant features and reduce the total number of dimensions and therefore the complexity of a given data set.

In their paper \enquote{A Tutorial on Principal Component Analysis} (\cite{Shlens, 2014}) Shlens gives an in-depth but applied approach into PCA. The discussed concepts and steps in classical PCA do reflect to the different quantum based PCA solutions proposed by the research community.

- Redundancy > Covariance Matrix > Diagonalize the covariance Matrix
=> classical preprocessing, classical prerpocessing, swap test [?]

- eigenvector decomposition
=> 

- singular value decomposition
=> 



Ref
- yanofsky paper
- qiskit textbook -> ref the relevant articles
- mooc
Shlens (2014): A Tutorial on Principal Component Analysis (arXiv: 14041100v1)
