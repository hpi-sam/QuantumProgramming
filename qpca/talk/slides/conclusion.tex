\begin{frame}[title-small={color=hpiorange, bg=none, text=Conclusion}]
	\maketitle
\end{frame}


\begin{frame}{Conclusion}
  \begin{minipage}{1.0\textwidth}
    \begin{itemize}
      \item different algorithms for qPCA have been proposed by research community $\rightarrow$ often additional constraints for practical implementing
      \begin{itemize}
        \item types of limitations discussed: number of qubits, qubit layout, circuit depth, circuit noise
      \end{itemize}
      \item integration: quantum state initialization is a non-trivial task when implementing and testing quantum algorithms
      \begin{itemize}
        \item three techniques discussed: classical register, density matrix, amplitude amplification
      \end{itemize}
      \item design: support through the different high-level libraries
      \begin{itemize}
        \item python cirq library discussed: circuit, operation, gate, value equality, testing
      \end{itemize}
    \end{itemize}
  \end{minipage}
\end{frame}


\begin{frame}{Conclusion}
  \begin{minipage}{1.0\textwidth}
    \begin{itemize}
      \item implementation: support through the different high-level libraries
      \begin{itemize}
        \item python unit-testing and end-to-end component testing discussed: classical assertions, circuit simulation
        \item not discussed: other types of testing for quantum algorithms, e.~g. projection based runtime assertions
        \item not discussed: error code correction to address noise or mixed states
      \end{itemize}
      \item maintenance: enhancements on the simple qPCA algorithm applied
      \begin{itemize}
        \item types of effects / changes discussed: preprocessing, post-processing, testing
      \end{itemize}
    \end{itemize}
  \end{minipage}
\end{frame}
